\documentclass[12pt,twoside,notitlepage]{article}
\usepackage{enumerate}
\usepackage{paralist}

\begin{document}

\section{Chapter 2}

\subsection{Notes}

Notes Here.

\subsection{Solutions}
\setdefaultleftmargin{30pt}{}{}{}{}{}
1.


\begin{enumerate}[(a)]
\setlength\itemindent{20pt}
\item With large n and small p, a less flexible method may not use the large 
  amount of observations available. A flexible method would be able to better 
  estimate the true f, just like Figure 2.3 -- 2.6. Caution must be taken not 
  to over-fit the data.
\item With small n and large p, a less flexible method may protect us from 
fluctuations in the observation due to small n. 
\item A nonlinear relationship between p and the response, may require a more 
  flexible method as in shown in Figure 2.11.
\item Highly flexible methods would be prone to over-fit ( i.e. fit the errors) 
\end{enumerate}
\setlength\itemindent{0pt}
2.
\begin{enumerate}[(a)]
\setlength\itemindent{20pt}
\item Since salaries are continuous, this is better analysed through a regression.
Seeking relationship between response and a given predictor : inference.  
n = 500. p = profit, number of employees, and industry.
\item Success or Failure is binary $\rightarrow$ category. Only interested in 
  outcome: Prediction. n = 20, p = price charged for the product, marketing 
  budget, competition price, and ten other variables.
\item Prediction of value: regression.  n = 3, p = U.S. market, British and German
  market.
\end{enumerate}  


\end{document}
